
\label{sect:conclusion}

We have shown that galaxy SED templates can be learned directly from a data set of broadband photometry.
Large sets of photometry at various redshifts can be leveraged to reconstruct high resolution features, such as the H$\alpha$, H$\beta$, H$\gamma$, H$\delta$, OII, and OIII emission lines, as well as Na and Mg absorption lines.
Simple post processing can further improve the resolution of these reconstructed lines.
The number of templates learned is variable and can be increased to more continuously sample the space of galaxy spectra and to improve photo-z results.

We used our templates to estimate photo-z's for a test set of galaxies using \bpz.
We found that training the standard set of templates that comes with \bpz\ decreases the fraction of bad photo-z's by 21\%, the bias by 63\% and the scatter by 23\%.
Our own trained naive templates yielded better results.
We learned a set of 20 templates from the data that reduced the fraction of bad photo-z's by 31\%, the bias by 91\%, and the scatter by 26\%.
These derived templates outperform the interpolated spectra used by BPZ.
The improvements in bias are almost sufficient to meet the requirements set for LSST, but another reduction by a factor of two is needed for the scatter.

The templates derived with our training algorithm demonstrate that accurate galaxy spectra can be learned from broadband photometry.
Our SED's could potentially be used for applications other than photo-z's, and our learning algorithm can be extended to other applications, such as learning supernova lightcurves from photometry.

Our derived templates and the code used to produce these results are publicly available in a dedicated Github repository: \url{https://github.com/dirac-institute/photoz_template_learning}.